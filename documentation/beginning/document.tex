\documentclass[a4paper,titlepage,11pt,twosides,floatssmall]{mwrep}
\usepackage[left=2.5cm,right=2.5cm,top=2.5cm,bottom=2.5cm]{geometry}
\usepackage[OT1]{fontenc}
\usepackage{polski}
\usepackage{amsmath}
\usepackage{amsfonts}
\usepackage{amssymb}
\usepackage{graphicx}
\usepackage{url}
\usepackage{tikz}
\usetikzlibrary{arrows,calc,decorations.markings,math,arrows.meta}
\usepackage{rotating}
\usepackage[percent]{overpic}
\usepackage[cp1250]{inputenc}
\usepackage{xcolor}
\usepackage{pgfplots}
\usetikzlibrary{pgfplots.groupplots}
\usepackage{listings}
\usepackage{matlab-prettifier}
\usepackage{enumitem,amssymb}
\definecolor{szary}{rgb}{0.95,0.95,0.95}
\usepackage{siunitx}
\usepackage{enumitem}
\setlist[enumerate]{itemsep=0mm}
\sisetup{detect-weight,exponent-product=\cdot,output-decimal-marker={,},per-mode=symbol,binary-units=true,range-phrase={-},range-units=single}
\SendSettingsToPgf
%konfiguracje pakietu listings
\lstset{
	backgroundcolor=\color{szary},
	frame=single,
	breaklines=true,
}
\lstdefinestyle{customlatex}{
	basicstyle=\footnotesize\ttfamily,
	%basicstyle=\small\ttfamily,
}
\lstdefinestyle{customc}{
	breaklines=true,
	frame=tb,
	language=C,
	xleftmargin=0pt,
	showstringspaces=false,
	basicstyle=\small\ttfamily,
	keywordstyle=\bfseries\color{green!40!black},
	commentstyle=\itshape\color{purple!40!black},
	identifierstyle=\color{blue},
	stringstyle=\color{orange},
}
\lstdefinestyle{custommatlab}{
	captionpos=t,
	breaklines=true,
	frame=tb,
	xleftmargin=0pt,
	language=matlab,
	showstringspaces=false,
	%basicstyle=\footnotesize\ttfamily,
	basicstyle=\scriptsize\ttfamily,
	keywordstyle=\bfseries\color{green!40!black},
	commentstyle=\itshape\color{purple!40!black},
	identifierstyle=\color{blue},
	stringstyle=\color{orange},
}

%wymiar tekstu (bez �ywej paginy)
\textwidth 160mm \textheight 247mm

%ustawienia pakietu pgfplots
\pgfplotsset{
	tick label style={font=\scriptsize},
	label style={font=\small},
	legend style={font=\small},
	title style={font=\small}
}

\def\figurename{Rys.}
\def\tablename{Tab.}

%konfiguracja liczby p�ywaj�cych element�w
\setcounter{topnumber}{0}%2
\setcounter{bottomnumber}{3}%1
\setcounter{totalnumber}{5}%3
\renewcommand{\textfraction}{0.01}%0.2
\renewcommand{\topfraction}{0.95}%0.7
\renewcommand{\bottomfraction}{0.95}%0.3
\renewcommand{\floatpagefraction}{0.35}%0.5

\begin{document}
%strona tytu�owa
\title{\bf Dokumentacja wst�pna\vskip 0.1cm}
\author{Jakub Wieczorek 277122}
\date{2018}

\makeatletter
\renewcommand{\maketitle}{
\begin{titlepage}
	\begin{center}{\LARGE {\bf
				Wydzia� Elektroniki i Technik Informacyjnych}}\\
		\vspace{0.4cm}
		{\LARGE {\bf Politechnika Warszawska}}\\
		\vspace{0.3cm}
	\end{center}
	\vspace{5cm}
	\begin{center}
		{\bf \LARGE Programowanie mikrokontroler�w w j�zyku C\\ zegar widmowy \vskip 0.1cm}
	\end{center}
	\vspace{1cm}
	\begin{center}
		{\bf \LARGE \@title}
	\end{center}
	\vspace{2cm}
	\begin{center}
		{\bf \Large \@author \par}
	\end{center}
	\vspace*{\stretch{6}}
	\begin{center}
		\bf{\large{Warszawa, \@date\vskip 0.1cm}}
	\end{center}
\end{titlepage}
}
\makeatother

\maketitle

%\tableofcontents
% !TEX encoding = cp1250
\chapter{Kr�tki opis projektu}

Ludzkie oko nie jest w stanie rozr�ni� migania od obrazu ciag�ego przy cz�stotliwo�ci wi�kszej ni� 20Hz. Zegar widmowy jest urz�dzeniem, kt�re rozkr�caj�c do wysokich obrot�w wskaz�wk�, na kt�rej s� zamontowane diody i w odpowidnich chwilach w��czaj�c je i wy��czaj�c wprowadza efekt ci�g�o�ci obrazu co pozwala wy�wietli� na nim r�ne ciekawe animacje. Decyzja o tym kiedy w��czy� i wy��czy� diody nale�y do mikrokontrolera  umieszczonego na wskaz�wce. Wskaz�wka zostanie przymocowana do silnika bldc z twardego dysku. Bardzo wa�n� cz�ci� projektu jest wyb�r sposobu zasilania. O ile sam silnik mo�e by� zasilony zasilaczem do telefonu lub laptopa to, kwestia zasilenia p�yty umieszczonej na wskaz�wce jest dosy� problematyczna. S� trzy sposoby z r�nymi wadami i zaletami:
\begin{enumerate}[label=\arabic*]
	\item \textbf{Zasilanie z baterii umieszczonej na wskaz�wce. Zalet� jest prostota, ale drastycznie zwi�kszy to mas� wskaz�wki;}
	\item \textbf{Zasilanie szczotkowe - generuje niepotrzebny ha�as;}
	\item \textbf{Zasilanie indukcyjne;}
\end{enumerate}
Wybra�em zasilanie indukcyjne, gdy� ma ono najwi�cej zalet i r�wnie� jest pewnym wyzwaniem konstrukcyjnym. Na wa� silnika zostanie nawini�ta cewka, na kt�rej zaciskach wytworzy si� napi�cie z powodu wiruj�cego pola magnetycznego. "�r�d�em" pola magnetycznego b�dzie kolejna cewka zamontowana na stojanie - jej zaciski b�d� przyczepione bezpo�rednio do baterii. Wyb�r animacji b�dzie mo�liwy za pomoc� aplikacji mobilnej, kt�ra z zegarem b�dzie komunikowa�a si� za pomoc� modu�u bluetooth. G��wn� animacj� b�dzie wy�wietlany czas i data. Mo�liwo�� wgrania kolejnych b�dzie mo�liwa za po�rednictwem karty SD. 
% !TEX encoding = cp1250
\let\clearpage\relax
\chapter{Wykaz realizowanych funkcji}
\begin{enumerate}[label=\arabic*]
	\item \textbf{Komunikacja bluetooth umo�liwiaj�ca zmiane wy�wietlanego obrazu za pomoc� dedykowanej aplikacji mobilnej;}
	\item \textbf{P�ynne, nie m�cz�ce oka wy�wietlanie obrazu i godziny - zastosowanie silnika wysokich obrot�w;}
	\item \textbf{Mo�liwo�� wgrania kolejnych animacji pobranych z dedykowanej biblioteki za pomoc� karty SD;}
	\item \textbf{W dalszych wersjach projektu przewiduje si�
	rozszerzenie funkcjonalno�ci poprzez dodanie czujnika temperatury i automatyczn� aktualizacj� godziny za pomoc� modu�u wifi (PSF-A85) pobieraj�cego informacje bezpo�rednio z sieci. W�wczas b�dzie mo�na do�o�y� animacje, kt�re wy�wietl� np. informacje o pogodzie;}
\end{enumerate}
\let\clearpage\relax
\chapter{Lista modu��w potrzebnych do realizacji projektu}

\begin{enumerate}[label=\arabic*]
	\item \textbf{P�ytka dwustronna z mikrokontrolerem stm32 zamontowana na wskaz�wce zegara;}
	\item \textbf{Diody rgb z modu�ami ws2812b po��czone szeregowo. Zredukuj� znacz�co liczbe pin�w potrzebnych do komunikacji z diodami oraz dadz� mo�liwo�� wy�wietlania kolorowych animacji (powstaje problem poboru dostatecznej ilo�ci pr�du z wyindukowanego zr�d�a napi�cia - diod b�dzie du�o);}
	\item \textbf{Przetwornik analogowo cyfrowy do pobrania informacji z czujnika Halla lub fotodiody;}
	\item \textbf{Modu� bluetooth XM-15B;}
	\item \textbf{Modu� komunikacyjny z kart� SD;}
\end{enumerate}
\let\clearpage\relax
\chapter{Zarys algorytmu}
Na obudowie urz�dzenia zostanie zamontowana fotodioda lub czujnik Halla, kt�ry b�dzie wykrywa� moment wykonania pe�nego obrotu przez wskaz�wk�. Podczas wykrycia przys�oni�cia fotodiody lub czujnika mikrokontroler w��czy timer, kt�ry zliczy milisekundy do czasu kolejnego oczytu z czujnika - jest to czas pe�nego obrotu wskaz�wki. Kolejnym krokiem b�dzie ustawienie przerwa� od timera na (czas~pe�nego~obrotu) / 360. Mikrokontroler w funkcji obs�ugi przerwania b�dzie aktualizowa� stan diod w zale�no�ci od wybranego obrazu.
\end{document}