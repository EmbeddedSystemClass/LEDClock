\subsection*{Punktacja}

Początek realizacji 08 .03.\+2018; Projekt wstępny 10.\+04.\+2018; Przekazanie zadań 05.\+06.\+2018; Wystawienie punktów 12.\+06.\+2018

\subsection*{Treść}

G\+B.\+S\+N.\+3 Baza danych C\+VE (\href{https://cve.mitre.org/}{\tt https\+://cve.\+mitre.\+org/}) gromadzi informacje o tzw. podatnościach, czyli błędach oprogramowania prowadzących do naruszenia bezpieczeństwa systemów. Podatności dotyczą zarówno oprogramowania uniwersalnego (np. serwera W\+WW Apache) jak i oprogramowania ściśle związanego z konkretnym sprzętem (np. firmware rutera Wi-\/\+Fi prod. D-\/\+Link model A\+C1200). W tym drugim przypadku podany jest zawsze producent i model sprzętu, baza nie zawiera jednak informacji o klasie sprzętu, nie wiemy więc czy podatność dotyczy np.\+: smartfona, urządzenia typu domowy ruter internetowy, sterownika przemysłowego, rutera klasy operatorskiej, itd. Baza C\+VE w opisie podatności zawiera zawsze U\+RL prowadzący do strony producenta. Naszym celem jest pozyskanie informacji o klasie sprzętu na podstawie analizy strony W\+WW producenta. Do tego celu należy wykorzystać sieć neuronową klasyfikującą strony na podstawie ograniczonego zbioru atrybutów z nich pozyskanych. Do przetwarzania stron należy użyć narzędzia Beautiful\+Soup – \href{https://www.crummy.com/software/BeautifulSoup/bs4/doc/}{\tt https\+://www.\+crummy.\+com/software/\+Beautiful\+Soup/bs4/doc/}. Zespół otrzyma dane treningowe oraz testowe. Uwaga\+: do tego zadania wymagane jest wykorzystanie języka Python.

\subsection*{Warunki zaliczenia i punktacja\+:}

Program powinien być napisany w języku Python; C/\+C++ lub Java. Preferowany jest język Python, w niektórych zadaniach Python jest wymagany. Preferowanym środowiskiem jest Unix, Windows jest dopuszczalne. Zaliczenie zadania odbywa się na podstawie projektu wstępnego (5p), sprawozdania (10p) oraz pokazu działającego programu i oceny kodu (15 p.). Pokaz w godz. konsultacji lub innym uzgodnionym terminie. Opóźnienie powoduje obniżenie punktacji. Niedostarczenie projektu wstępnego i/lub sprawozdania (opóźnienie 2 tyg. w dostarczeniu projektu wstępnego lub niedostarczenie w terminie ostatecznym) powoduje niezaliczenie zadania i przedmiotu.

\subsubsection*{Co powinien zawierać projekt wstępny (3-\/4 strony)\+:}

Sprawozdanie musi być dostarczone osobiście -\/ “na papierze”, postać elektroniczna nie jest akceptowana i nie będzie brana pod uwagę. Dostarczone sprawozdanie może, ale nie musi być od razu omówione, ocena ze sprawozdania wstępnego zostanie wystawiona w ciągu tygodnia od dostarczenia.

Sprawozdanie wstępne powinno zawierać\+:
\begin{DoxyEnumerate}
\item Skład zespołu, treść zadania z wyraźnym oznaczeniem wariantu (o ile dotyczy) – najlepiej na samym początku w tytule;
\item interpretację treści zadania – dodatkowe założenia, doprecyzowanie treści, itp.
\item krótki opis funkcjonalny – “black-\/box” (bez informacji o implementacji) opis i uzasadnienie przyjętego rozwiązania przyjętego sposobu realizacji (np. dobór typu sieci neuronowej i sposobu jej symulacji, itd).
\item planowany podział na komponenty i sposób komunikacji między nimi (np. specyfikacja kluczowego A\+PI)
\item zarys koncepcji implementacji (najważniejsze funkcje, algorytmy, obiekty komunikacyjne, itp.).
\end{DoxyEnumerate}

Co powinien zawierać projekt ostateczny (6-\/10 stron)
\begin{DoxyEnumerate}
\item To co projekt wstepny, ewentualnie poprawione i uzupełnione;
\item pełen opis funkcjonalny ;
\item opis interfejsu użytkownika;
\item postać plików konfiguracyjnych, logów, itp.;
\item raport z przeprowadzonych testów oraz wnioski;
\item opis wykorzystanych narzędzi, bibliotek, itp; 7 opis metodyki testów i wyników testowania;
\end{DoxyEnumerate}

\subsubsection*{W\+AŻ\+NE\+:}

drukowane sprawozdanie powinno składać się ze zszytych (trwale zespolonych kartek), sprawozdania składające się z luźnych kartek lub kartek spiętych spinaczem biurowym nie będą przyjmowane \+:-\/(

Uwaga\+:
\begin{DoxyEnumerate}
\item Sprawozdanie wstępne przekazane zostaje w formie pisemnej, nie jest wymagana obecność całego zespołu przy jego przekazaniu.
\item Konsultacje w trakcie realizacji projektu nie wymagają obecności całego zespołu.
\item Sprawozdanie końcowe i pokaz funkcjonowania musi odbywać się w obecności całego zespołu.
\item opisywanie ogólnie znanych rozwiązań jest niepotrzebne i niewskazane
\item Po pokazie działania i przekazaniu dokumentacji kod źródłowy proszę przesyłać wyłącznie w formacie tar.\+gz lub zip na adres\+: \href{mailto:GBlinowski@ii.pw.edu.pl}{\tt G\+Blinowski@ii.\+pw.\+edu.\+pl}
\end{DoxyEnumerate}

U\+W\+A\+GA dopiero po zaaprobowaniu programu i sprawozdania! Załącznik powinien mieć nazwę wg. wzorca\+: P\+S\+Z\+T\+\_\+2018\+L\+\_\+\+A\+Iksinsnski.\+tar.\+gz 